\documentclass[journal,onecolumn]{IEEEtran}
\usepackage{cite}
\usepackage{amsmath,amssymb,amsfonts}
\usepackage{algorithmic}
\usepackage{graphicx,subfigure}
\graphicspath{{figures/}}
\usepackage{textcomp}
\usepackage{xcolor}
\usepackage{cleveref}
\hyphenation{op-tical net-works semi-conduc-tor}
	\begin{document}
		
		\title{Paper Title* (use style: paper title)}
		
		\author{line 1: 1st Given Name Surname \\line 2: dept. name of organization 
			(of Affiliation)\\line 3: name of organization 
			(of Affiliation)\\line 4: City, Country\\line 5: email address or ORCID}

	\maketitle
	
\begin{abstract}
\textbf{This electronic document is a “live” template and already defines the components of your paper [title, text, heads, etc.] in its style sheet.  *CRITICAL:  Do Not Use Symbols, Special Characters, Footnotes, or Math in Paper Title or Abstract.} 
\end{abstract}
	

\begin{IEEEkeywords}
\textbf{component, formatting, style, styling, insert (key)}
\end{IEEEkeywords}
	
	\IEEEpeerreviewmaketitle
	
\section{Introduction}
This template, modified in MS Word 2007 and saved as a “Word 97-2003 Document” for the PC, provides authors with most of the formatting specifications needed for preparing electronic versions of their papers. All standard paper components have been specified for three reasons:  ease of use when formatting individual papers, automatic compliance to electronic requirements that facilitate the concurrent or later production of electronic products, and  conformity of style throughout a conference proceedings. Margins, column widths, line spacing, and type styles are built-in; examples of the type styles are provided throughout this document and are identified in italic type, within parentheses, following the example\cite{IEEEhowto:kopka}. Some components, such as multi-leveled equations, graphics, and tables are not prescribed, although the various table text styles are provided \cite{Simpson,I. S. Jacobs,Y. Yorozu} . The for matter will need to create these components, incorporating the applicable criteria that follow.

\section{EASE OF USE}
\subsection{Selecting a Template (Heading 2)}
First, confirm that you have the correct template for your paper size. This template has been tailored for output on the A4 paper size. If you are using US letter-sized paper, please close this file and download the Microsoft Word, Letter file.	
\subsection{Maintaining the Integrity of the Specifications}
The template is used to format your paper and style the text. All margins, column widths, line spaces, and text fonts are prescribed; please do not alter them. You may note peculiarities. For example, the head margin in this template measures proportionately more than is customary. This measurement and others are deliberate, using specifications that anticipate your paper as one part of the entire proceedings,
\section{PREPARE YOUR PAPER BEFORE STYLING}
Before you begin to format your paper, first write and save the content as a separate text file. Complete all content and organizational editing before formatting. Please note sections A-D below for more information on proofreading, spelling and grammar.
Keep your text and graphic files separate until after the text has been formatted and styled. Do not use hard tabs, and limit use of hard returns to only one return at the end of a paragraph. Do not add any kind of pagination anywhere in the paper. Do not number text heads-the template will do that for you.
\subsection{Abbreviations and Acronyms}
Define abbreviations and acronyms the first time they are used in the text, even after they have been defined in the abstract. Abbreviations such as IEEE, SI, MKS, CGS, sc, dc, and rms do not have to be defined. Do not use abbreviations in the title or heads unless they are unavoidable.
\subsection{Units}
\begin{itemize}
	\item Use either SI (MKS) or CGS as primary units. (SI units are encouraged.) English units may be used as secondary units (in parentheses). An exception would be the use of English units as identifiers in trade, such as “3.5-inch disk drive”.
	\item Avoid combining SI and CGS units, such as current in amperes and magnetic field in oersteds. This often leads to confusion because equations do not balance dimensionally. If you must use mixed units, clearly state the units for each quantity that you use in an equation.
	\item Do not mix complete spellings and abbreviations of units: “Wb/m2” or “webers per square meter”, not “webers/m2”.  Spell out units when they appear in text: “. . . a few henries”, not “. . . a few H”. Identify applicable funding agency here. If none, delete this text box.
	\item  Use a zero before decimal points: “0.25”, not “.25”. Use “cm3”, not “cc”. (bullet list)
\end{itemize}
\section{USING THE TEMPLATE}
After the text edit has been completed, the paper is ready for the template. Duplicate the template file by using the Save As command, and use the naming convention prescribed by your conference for the name of your paper. In this newly created file, highlight all of the contents and import your prepared text file. You are now ready to style your paper; use the scroll down window on the left of the MS Word Formatting toolbar.
\section{Conclusion}
Number footnotes separately in superscripts. Place the actual footnote at the bottom of the column in which it was cited. Do not put footnotes in the abstract or reference list. Use letters for table footnotes.
	
\begin{thebibliography}{1}
		
\bibitem{IEEEhowto:kopka}
H.~Kopka and P.~W. Daly, \emph{A Guide to \LaTeX}, 3rd~ed.\hskip 1em plus
0.5em minus 0.4em\relax Harlow, England: Addison-Wesley, 1999.

\bibitem{Simpson} Homer J. Simpson. \textsl{Mmmmm...donuts}. Evergreen Terrace Printing Co., Springfield, SomewhereUSA, 1998

\bibitem{I. S. Jacobs} I. S. Jacobs and C. P. Bean, “Fine particles, thin films and exchange anisotropy,” in Magnetism, vol. III, G. T. Rado and H. Suhl, Eds. New York: Academic, 1963, pp. 271–350.

\bibitem{Y. Yorozu} Y. Yorozu, M. Hirano, K. Oka, and Y. Tagawa, “Electron spectroscopy studies on magneto-optical media and plastic substrate interface,” IEEE Transl. J. Magn. Japan, vol. 2, pp. 740–741, August 1987 [Digests 9th Annual Conf. Magnetics Japan, p. 301, 1982].
		
\end{thebibliography}

\end{document}